\documentclass{report}
\usepackage[spanish]{babel}
\usepackage[utf8]{inputenc}
\usepackage{algorithmicx}
\usepackage{hyperref}
\usepackage{listings}

\newcommand{\pythonprompt}{\textgreater\textgreater\textgreater}
\lstset{numbers=left}
\title{El lenguaje Python}
\author{Sergio Hernández}
\date{}
\begin{document}
\maketitle

En esta sesión se revisarán los aspectos computacionales más básicos de Python y
por qué es la decisión de una clase de sistemas complejos usarlo. 

Dentro de los detalles técnicos se verán aspectos fundamentales de 
los lenguajes de programación particularmente apliados a Python como
la asignación de variables, la definición de funciones, el funcionamiento
del {\tt{for}} y el funcionamiento del {\tt{if}}.

\section{Python y otros lenguajes}

Python es un lenguaje de varios paradigmas, entre ellos es un lenguaje 
interpretado, orientado a objetos y funcional. Otro aspecto 
importante de Python es que es un lenguaje de alto nivel \cite{wikipython, 
httlcs}

Este, como otros lenguajes de computación, son {\it{Turing completos}} lo que
significa que todos ellos pertenecen a una clase de equivalencia en la que todas
las funciones que puede computar uno, la puede computar cualquier otro lenguaje
que comparte esta clasificación. O sea, Python puede computar lo mismo que C o
que Java o que el juego de la vida, entonces ¿por qué usar Python sobre otro 
lenguaje?

Hay dos razones para esto. La primera es que es su popularidad ha alcanzado una
masa crítica de desarrolladores los cuales conforman una comunidad bastante 
madura a la que se le pueden consultar problemas muy particulares, muchos de
ellos son de tipo científico. 

Otra razón que puede ser una explicación para la popularidad antes mencionada
es la filosofía detrás de Python. El contraste sobre otros lenguajes es que el 
código hecho en Python debe ser legible y sencillo, sin embargo no lacónico
y enmarañado como lo es el {\it{código espagueti}}\footnote{\url{https://en.wikipedia.org/wiki/Spaghetti_code}} 
en el que se puede caer muy fácilmente con lenguajes como C ó Java. Python 
incorpora dentro de su sintaxis aspectos que en otros lenguajes son consejos
y buenas prácticas de desarrollo. Una función para mostrar las diferencias 
entre Python y C puede ser la función factorial
\[
n! = \left\{
            \begin{array}{ll}
                    1              & \mbox{si } n = 0 \\
                    n\cdot(n-1)! & \mbox{si } n > 0
            \end{array}
     \right.
\]

La implementación de esta función en C puede ser como sigue:

\lstinputlisting[language=C, firstline=1]{factorial.c}

Analizando este código podemos ver aspectos que caracterizan, en este caso, a
la programación en C. Primero, todas las líneas deben terminar con el símbolo
{\tt{;}} para indicarle al compilador donde acaba una línea, otro aspecto 
distintivo de C es el hecho de que es un lenguaje tipado, es decir, debe 
specificarse a que tipo de dato corresponde cada variable a usar, por ejemplo 
en las líneas 5, 7, 8, 12 y 14. 

Pero otro par de aspectos de sintaxis fundamentales son los siguientes. 
En la línea 7, se puede ver que no está identada, es decir, no se uso la 
popular práctica de programación de usar un tabulador para indicar todas las 
líneas que pertenecen a un mismo bloque de código. Lo importante es que al ser
una práctica de programación, el compilador no lanza ningún error por no estar
identada, simplemente es una petición para procurar que el código sea más 
legible. El otro aspecto lo podemos ver en la línea 8 donde podemos ver que
la instrucción {\tt{for}} inicializa una variable, verifica si la condición
de escape no se ha cumplido y en la sección donde se reevalúa el contador 
se hace una operación al mismo tiempo sobre la variable resultado. Este tipo
de prácticas puede hacer el código particularmente difícil de seguir una
vez que se rebasa un umbral de líneas.

Para poder correr el programa primero hay que compilarlo, algo que caracteriza
a C:

\begin{verbatim}
$ gcc factorial.c -std=gnu99 -o factorial
$ ./factorial
\end{verbatim}

En cambio en Python, al ser un programa interpretado (lo que significa que 
podemos acceder al intérprete y ejecutar línea por línea) recibe un 
{\it{script}} describiendo el programa. Ese programa es el siguiente

\lstinputlisting[language=Python, firstline=1]{fact1.py}

En este código podemos ver que ya no se usan llaves para indicar las 
instrucciones de una función o de un ciclo for, en vez de eso se ocupa 
el tabulador para indicar todas las líneas que pertenecen a un bloque de 
código. Por ejemplo las líneas 6 a la 9 pertenecen a la definición de la línea
5 y de la 12 a la 14 pertenecen al bloque definido por la línea 11. Es decir,
el desarrollo de Python incluyó como necesario el uso de tabuladores para
especificar bloques de código; lo que antes era una buena práctica se convirtió
en una necesidad de la sintaxis.

Como se puede ver en el código exhibido está el hecho que no es necesario 
definir un tipo específico para cada variable ni es necesario usar {\tt{;}}
para especificar el final de la línea. 

La filosofía de Python puede verse desde el intérprete como se muestra a 
continuación:

\begin{verbatim}
$ python
Python 2.7.9 (default, Jun 29 2016, 13:08:31) 
[GCC 4.9.2] on linux2
Type "help", "copyright", "credits" or "license" for more information.
>>> import this
The Zen of Python, by Tim Peters

Beautiful is better than ugly.
Explicit is better than implicit.
Simple is better than complex.
Complex is better than complicated.
Flat is better than nested.
Sparse is better than dense.
Readability counts.
Special cases aren't special enough to break the rules.
Although practicality beats purity.
Errors should never pass silently.
Unless explicitly silenced.
In the face of ambiguity, refuse the temptation to guess.
There should be one-- and preferably only one --obvious way to do it.
Although that way may not be obvious at first unless you're Dutch.
Now is better than never.
Although never is often better than *right* now.
If the implementation is hard to explain, it's a bad idea.
If the implementation is easy to explain, it may be a good idea.
Namespaces are one honking great idea -- let's do more of those!
>>> 
\end{verbatim}

Otro aspecto que caracteriza a Python es el uso de listas. Una lista es un
tipo de datos que puede contener más objetos (números, letras, otras listas).
Al pensar en tener datos en listas podemos pensar en un algoritmo que haga uso
de esta característica. La implementación la podemos ver en el siguiente 
segmento de código:

\lstinputlisting[language=Python, firstline=1]{fact2.py}

\subsection{Entrando y saliendo de Python}

Para usar el intérprete de Python existe la distribución Anaconda\footnote{
\url{https://www.continuum.io/downloads}} para todos los usuario de Windows. 
Para los usuario de Linux, todas las distribuciones ya incluyen Python como 
parte integral de la distribución. Una vez instalado, desde la línea de 
comandos en ambos sistemas operativos podemos llamar a Python de la siguiente
forma:

\begin{verbatim}
$ python
Python 2.7.9 (default, Jun 29 2016, 13:08:31) 
[GCC 4.9.2] on linux2
Type "help", "copyright", "credits" or "license" for more information.
>>> 
\end{verbatim}

El comando desde linux (o windows) se llama python y una vez ejecutado se
debe mostrar el indicador {\tt{\pythonprompt}} que señala estar a la espera de 
recibir instrucciones para el lenguaje. Como una nota técnica importante hay 
que decir que estaremos usando, todavía, la versión 2.7 de Python en vez de la 
versión 3.5. También estaremos usando el intérprete {\tt{ipython}} que hace 
más sencillo el uso del lenguaje a través del lenguaje.

\begin{verbatim}
$ ipython
Python 2.7.9 (default, Jun 29 2016, 13:08:31) 
Type "copyright", "credits" or "license" for more information.

IPython 5.1.0 -- An enhanced Interactive Python.
?         -> Introduction and overview of IPython's features.
%quickref -> Quick reference.
help      -> Python's own help system.
object?   -> Details about 'object', use 'object??' for extra details.

In [1]: 
\end{verbatim}

\section{Asignación de variables}

Una variable, igual que en cualquier otro lenguaje, es un símbolo que guarda
un valor especificado. Estos valores pueden ser cualquier cosa desde un número,
cadena de letras (o simplemente cadena, a partir de la traducción del inglés 
{\it{string}}), otra lista.

Para instanciar una variable hay que elegir un símbolo que debe empezar con una
letra y después indicar que valor va a representar. El siguiente diagrama 
ilustra esto.

\begin{algorithmic}
\State $a \gets 5$
\end{algorithmic}

En este caso estamos instanciando, lógicamente, a la variable a con el valor 5. En Python esto se hace de la siguiente forma:

\begin{verbatim}
>>> a = 5
>>> a
5
>>> 
\end{verbatim}

En este caso primero guardamos el {\tt{5}} en la variable {\tt{a}} y para
examinar su valor podemos, simplemente, teclear el nombre de la variable y dar
enter.

Otro ejemplo de asignación es el que sigue

\begin{algorithmic}
\State $b \gets ``hola"$
\end{algorithmic}

En este caso primero guardamos la cadena {\tt{hola}} en la variable {\tt{b}} y 
para examinar su valor podemos, simplemente, teclear el nombre de la variable 
y dar enter. En Python:

\begin{verbatim}
>>> b = "hola"
>>> b
'hola'
>>> 
\end{verbatim}

En este caso hay que notar dos cosas. En la asignación estamos usando comillas 
dobles lo que significa que la variable {\tt{b}} va a guardar la secuencia de
letras {\tt{hola}}. Al imprimirla podemos ver que lo que se imprime en pantalla
es la leyenda {\tt{'hola'}} encapsulada por unas comillas simples. Para 
ilustrar la diferencia entre una cadena y una variable numérica podemos hacer
el siguiente ejercicio para el cual requerimos hacer una suma entre números 
primero, esto se hace de la siguiente manera

\begin{verbatim}
>>> 5+55
60
>>> 
\end{verbatim}

Ahora que hemos visto como funciona la suma hagamos el siguiente ejercicio que 
consta de dos partes, en la primera le asignaremos {\tt{55}} a la variable 
{\tt{a}} y le sumaremos un {\tt{5}}. En la segunda parte guardaremos la cadena
{\tt{55}} y le sumaremos el valor {\tt{5}} para observar el resultado de cada
parte, una después de la otra. En ambos casos primero examineremos el valor
de la variable {\tt{a}}

\begin{verbatim}
>>> a=55
>>> a
55
>>> a+5
60
>>> a="55"
>>> a
'55'
>>> a+5
Traceback (most recent call last):
  File "<stdin>", line 1, in <module>
TypeError: cannot concatenate 'str' and 'int' objects
\end{verbatim}

El hecho de que en la última parte del ejercicio este fallara es porque
quisimo sumar un 5 a una cadena de texto y el lenguaje de programación no 
tiene definido como hacer esta operación (veremos más adelante que para la
multiplicación de un texto y un número sí existe una definición explícita).
Es decir, aunque para nosotros la cadena {\tt{'55'}} se lee como un {\tt{55}}
al que sabemos sumarle un número {\tt{5}}, el lenguaje no lo sabe, es decir,
es otro tipo de dato distinto a los números para los cuales sí está definida
la operación de suma.

Otra cosa que podmeos ver es el hecho de que podemos guardar valores de tipo 
distintos a una misma variable sin tener que realizar ninguna otra 
redefinición.

Si queremos guardar una lista tendremos que guardar los elementos que la 
componen separados por comas ({\tt{,}}) y encerrados entre corchetes cuadrados
({\tt{[]}}). 

\begin{verbatim}
>>> L = [1,2,3,4]
>>> L
[1, 2, 3, 4]
>>> 
\end{verbatim}

La lista {\tt{L}} solo guarda números, si queremos que guarde valores más 
heterogéneos, por ejemplo, un número, una cadena y una lista lo podemos hacer
de la siguiente forma.

\begin{verbatim}
>>> L = [10, "python", [5,6,7]]
>>> L
[10, 'python', [5,6,7]]
>>> 
\end{verbatim}

Si queremos usar el alguno de estos valores guardados solo tenemos que usar
el nombre de la lista seguida de unos corchetes cuadrados encerrando el índice
del elemento a recuperar empezando desde {\tt{0}}. Por ejemplo

\begin{verbatim}
>>> L[1]
'python'
>>>
>>> L[0]
10
>>>
\end{verbatim}

O bien podemos asignar alguno de esos valores a otra variable

\begin{verbatim}
>>> c = L[2]
>>> c
[5, 6, 7]
>>>
\end{verbatim}

Los siguientes son ejemplos fallidos para nombrar una variable

\begin{verbatim}
>>> 5a = 3
  File "<stdin>", line 1
    5a = 3
     ^
SyntaxError: invalid syntax
>>> .a = 3
  File "<stdin>", line 1
    .a = 3
    ^
\end{verbatim}

\section{Iteración}

\section{Condicionales}

\begin{thebibliography}{9}

\bibitem{wikipython}
Python (programming language). (2017, February 16). In Wikipedia, The Free
Encyclopedia. Retrieved 19:43, February 17, 2017, from 
\url{https://en.wikipedia.org/w/index.php?title=Python_(programming_language)&oldid=765828419}

\bibitem{httlcs}
Elkner J., Downy A. B., Meyer Chris M. (2017, Febrero 17). 
How to think like a computer scientist: Learning with Python. 
Recuperado Febrero 17, 2017 de 
\url{http://openbookproject.net/thinkcs/python/english2e/}. 

\end{thebibliography}
\end{document}
